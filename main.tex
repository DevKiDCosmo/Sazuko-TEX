\documentclass{sazuko}
\usepackage{graphicx} % Required for inserting images
\usepackage{lipsum}
\usepackage{amsmath}
\usepackage{amssymb}

\title{New Article Template -- Sazuko}
\author{Duy Nam Schlitz}
\date{December 2024}
\SZKundertitle{Something new}

\SZKyear{2024}
\SZKlocation{Hanau}
\SZKdates{03.12}
\SZKcopyright{2024}{Duy Nam Schlitz}
\SZKarchive{123948}


\abstract{\lipsum[2]}

\begin{document}

\maketitle

\tableofcontents

\newpage

\section{Two--Column Section}
This section starts in the default layout (two--column).

\lipsum[1-12]

\togglelayout

\section{One--Column Section}
This section switches to one--column layout, starting on a new page.

\subsection{Somethings}
\subsection{Somethings}
\subsection{Somethings}


\lipsum[1-4]\footnote{\lipsum[1]}

$$ \frac{1}{|A|} \int_A f(x) dx + \frac{1}{|B|} \int_B
f(x) dx = \int_X f(x) dx $$

\togglelayout

\importantsection{Back to Two--Column Layout}
This section is in two--column layout, starting on a new page. \lipsum[1-9]

\subsection{Hello}

$$ \frac{1}{|A|} \int_A f(x) dx + \frac{1}{|B|} \int_B
f(x) dx = \int_X f(x) dx $$

\importantsection{Another Custom Section}
This section has a blue headline in two--column mode.\footnote{\lipsum[1-2]}

\tipsection{Tip}
\randomformat{Something is not right here. You can change the proportion.}

\lipsum[1-5]

\togglelayout

\section{Seifert-van Kampen Theorem}

\textbf{Proof}

Let $\varphi: A \to [0, 1]$ and $\psi: B \to [0, 1]$
be two continuous functions such that $\varphi(a) = \psi(b)$ for all $a \in A$, $b \in B$. Define a new
function $F: X \to [0, 1]$ by

$$F(x) = \begin{cases} \varphi(x) & x \in A \\
\psi(x) & x \in B \end{cases}$$

Since $\mathcal{C}$ is a collection of open subsets
of $X$, we can define a new function $f: X \to \mathbb{R}$ by

$$f(x) = F(\varphi^{-1}(x))$$

where $\varphi^{-1}: [0, 1] \to A$ is the inverse
function of $\varphi$. Similarly, we can define
another function $g: X \to \mathbb{R}$ by

$$g(x) = F(\psi^{-1}(x))$$

where $\psi^{-1}: [0, 1] \to B$ is the inverse
function of $\psi$.

By construction, we have

$$f(a) = g(b)$$

for all $a \in A$, $b \in B$. Therefore, we can write

$$\int_A f(x) dx + \int_B g(x) dx = \int_X f(x) dx$$

Since $\varphi$ and $\psi$ are continuous functions,
we have

$$\frac{1}{|A|} \int_A f(x) dx + \frac{1}{|B|} \int_B
f(x) dx = F(\varphi(a))$$

for all $a \in A$. Similarly, we have

$$F(\psi(b)) = \frac{1}{|A|} \int_A f(x) dx + \frac{1}{|B|} \int_B f(x) dx$$

for all $b \in B$.

Since $F(x)$ is a continuous function that takes values in $[0, 1]$, we
have

$$\lim_{x \to y} F(\varphi^{-1}(x)) = F(\psi^{-1}(y))$$

for all $x, y \in [0, 1]$. Therefore, we can write

$$F(\varphi(a)) + F(\psi(b)) = F(\varphi^{-1}(b)) + F(\psi^{-1}(a))$$

for all $a \in A$, $b \in B$.

Substituting the previous equations, we get

$$\frac{1}{|A|} \int_A f(x) dx + \frac{1}{|B|} \int_B g(x) dx =
F(\varphi^{-1}(b)) + F(\psi^{-1}(a))$$

Since $F(x)$ is a continuous function that takes values in $[0, 1]$, we
have

$$\lim_{x \to y} F(x) = F(y)$$

for all $x, y \in [0, 1]$. Therefore, we can write

$$\frac{1}{|A|} \int_A f(x) dx + \frac{1}{|B|} \int_B g(x) dx =
F(\varphi^{-1}(b))$$

for all $b \in [0, 1]$. Similarly, we have

$$F(\psi^{-1}(a)) = \frac{1}{|A|} \int_A f(x) dx + \frac{1}{|B|} \int_B
g(x) dx$$

for all $a \in [0, 1]$.

Combining the two equations, we get

$$\frac{1}{|A|} \int_A f(x) dx + \frac{1}{|B|} \int_B g(x) dx =
F(\varphi^{-1}(b)) + F(\psi^{-1}(a))$$

for all $a \in [0, 1]$, $b \in [0, 1]$.

Since $F(x)$ is a continuous function that takes values in $[0, 1]$, we
have

$$\lim_{x \to y} F(\varphi^{-1}(x)) = F(\psi^{-1}(y))$$

for all $x, y \in [0, 1]$. Therefore, we can write

$$\frac{1}{|A|} \int_A f(x) dx + \frac{1}{|B|} \int_B g(x) dx =
F(\varphi^{-1}(b))$$

for all $b \in [0, 1]$. Similarly, we have

$$F(\psi^{-1}(a)) = \frac{1}{|A|} \int_A f(x) dx + \frac{1}{|B|} \int_B
g(x) dx$$

for all $a \in [0, 1]$.

Combining the two equations, we get

$$\lim_{x \to y} F(\varphi^{-1}(x)) = \lim_{y \to x} F(\psi^{-1}(y))$$

for all $x, y \in [0, 1]$. Therefore, we can write

$$F(x) = \int_X f(y) dy$$

for all $x \in X$.

Substituting the equation for $f$, we get

$$\frac{1}{|A|} \int_A F(\varphi^{-1}(y)) dy + \frac{1}{|B|} \int_B
F(\psi^{-1}(y)) dy = \int_X f(y) dy$$

Since $\varphi$ and $\psi$ are continuous functions, we have

$$\lim_{y \to x} F(\varphi^{-1}(y)) = F(x)$$

for all $x \in A$. Therefore, we can write

$$\frac{1}{|A|} \int_A f(y) dy + \frac{1}{|B|} \int_B g(y) dy = \int_X
f(y) dy$$

for all $y \in X$.
This proves the Seifert-van Kampen theorem.

\subsubsection{Sometimes}

\togglelayout

\tipsection{Tip}
\lipsum[1]

\importantsection{Important / Definition / Key Point}
\lipsum[1]
\warningsection{Warning}
\lipsum[1]
\examplesection{Example}
\lipsum[1]
\resultsection{Result}
\lipsum[1]
\questionsection{What is this?/Exercise Part}
\lipsum[1]

\section{New}

\definitionAN{Math}{Is hArd.}{Please learn it}


\definition{Math}{Is hArd.}

\exerciseWS{Hard}{Exercise}{Solution and Explanation}
\exercise{Something}{Exercise}

\introduction{DOn't even try}{No one}

\summary{Anjin}{Is strange}
% 안진

\end{document}